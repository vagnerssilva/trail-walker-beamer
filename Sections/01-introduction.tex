%*----------- SLIDE -------------------------------------------------------------
\begin{frame}[t]{Introdução} 
    \transdissolve[duration=0.5]

    Sobre o orientador
    %\newline
        \begin{columns}[t]
            \column{.05\linewidth}
            \column{1\linewidth}
                \begin{enumerate}
                    \item Graduado em Engenharia Elétrica pela UFPR e Mestre em Engenharia de Produção pela UFSC
                    \item Pesquisador do Instituto Brasileiro de Robótica, uma ação conjunta entre o Senai Cimatec e o Centro Alemão de Inteligência Artificial
                    \item Professor convidado dos cursos de especialização em Automação, Controle e Robótica, e de Sistemas Elétricos de Potência do Senai CIMATEC
                \end{enumerate}
            \column{.6\linewidth}
        \end{columns}
%*----------- notes
    \note[item]{Notes can help you to remember important information. Turn on the notes option.}
\end{frame}
%-
%*----------- SLIDE -------------------------------------------------------------
\begin{frame}[t]{Justificativa} 
    \transdissolve[duration=0.5]
    %\newline
        \begin{columns}[t]
            \column{.05\linewidth}
            \column{1\linewidth}
                \begin{enumerate}
                    \item Robôs antropomórficos podem ser amplamente utilizados em diversas áreas do dia-a-dia
                    \item Robôs antropomórficos possuem uma das configurações mais eficiente para locomoção de ambiente de difícil navegação
                \end{enumerate}
            \column{.6\linewidth}
        \end{columns}
%*----------- notes
    \note[item]{Notes can help you to remember important information. Turn on the notes option.}
\end{frame}
%-
%*----------- SLIDE -------------------------------------------------------------
\begin{frame}[c]{Problema de pesquisa} 
    \transdissolve[duration=0.5]
   
    \begin{center}
        \Wider{%
        \begin{shaded}
        \begin{center}
            \vspace*{0.4cm}
            \resizebox{!}{1.3cm}{%
               % \color{bg} O objetivo é ter um objetivo.
                \begin{tabular}{ccc}
                    De que forma um sistema mecânico pode \\
                    se locomover de tal forma a desviar-se \\
                    de obstáculos?
                    % De que forma garantir a segurança humana em ambientes \\
                    % confinados e de difícil acesso através da ajuda de \\ 
                    % robôs bípedes?       
                  \end{tabular}
            }%
        \end{center}
        \end{shaded}
        }%
    \end{center}
    
   
%*----------- notes
    \note[item]{Notes can help you to remember important information. Turn on the notes option.}
\end{frame}
%-
%*----------- SLIDE -------------------------------------------------------------
\begin{frame}[t]{Objetivos} 
    \transdissolve[duration=0.5]
    
        Desenvolver um robô de pequeno porte que se desloca sobre dois pés. O robô deve ser capaz de se locomover e desviar de obstáculos em um determinado ambiente.
        \newline
        \newline
         Específicos
        \begin{columns}[t]
            \column{.15\linewidth}
            \column{1\linewidth}
                \begin{enumerate}
                    \item Desenvolver algoritmos utilizando o ROS2
                    \item Implementar visão computacional para a navegação
                    \item Simular o sistema robótico
                    \item Implementar principais funcionalidades de um humanóide
                    \item Realizar demonstração do sistema
                    \item Desenvolver artigos científicos
                \end{enumerate}
            \column{.6\linewidth}
        \end{columns}
%*----------- notes
    \note[item]{Notes can help you to remember important information. Turn on the notes option.}
\end{frame}
%-
%*----------- SLIDE -------------------------------------------------------------
% \begin{frame}[c]{Roadmap}
%     %\cutpic{0.30cm}{3cm}{element}
%     \begin{tabular}{cccc}
%         \rule{30pt}{0ex}  &   \includegraphics[width=.2\textwidth ]{element.png} & \rule{15pt}{0ex} \includegraphics[width=.15\textwidth]{notion.png} \rule{15pt}{0ex}& \includegraphics[width=.16\textwidth]{github.png}\\
%     \end{tabular}

%     \begin{tabular}{ccc}
%         \phantom{The text is invisible} &   \includegraphics[width=.2\textwidth]{trello.png} \rule{5pt}{0ex}& \includegraphics[width=.2\textwidth]{projectlibre.png} \\
%     \end{tabular}
% %*----------- notes
%     \note[item]{Notes can help you to remember important information. Turn on the notes option.}
% \end{frame}
%-
%*----------- SLIDE -------------------------------------------------------------
\begin{frame}[t]{Referencial teórico} 
    \transdissolve[duration=0.5]
    %\newline
        \begin{columns}[t]
            \column{.05\linewidth}
            \column{1\linewidth}
                \begin{itemize}
                    \item GUEDES, Vânia LS; BORSCHIVER, Suzana. Bibliometria: uma ferramenta estatística para a gestão da informação e do conhecimento, em sistemas de informação, de comunicação e de avaliação científica e tecnológica. Encontro Nacional de Ciência da Informação, v. 6, n. 1, p. 18, 2005.
                    \item PASCHOARELLI, Luis Carlos; MEDOLA, Fausto Orsi; BONFIM, Gabriel Henrique Cruz. Características Qualitativas, Quantitativas e Quali-quantitativas de Abordagens Científicas: estudos de caso na subárea do Design Ergonômico. Revista de Design, Tecnologia e Sociedade, v. 2, n. 1, p. 65-78, 2015.
                \end{itemize}
            \column{.6\linewidth}
        \end{columns}
%*----------- notes
    \note[item]{Notes can help you to remember important information. Turn on the notes option.}
\end{frame}
%-
%*----------- SLIDE -------------------------------------------------------------
\begin{frame}[c]{Metodologia}
    \begin{figure}
            \fbox{\includegraphics[scale=0.28]{metodologia.drawio.png}}
    \end{figure}
%*----------- notes
    \note[item]{Notes can help you to remember important information. Turn on the notes option.}
\end{frame}
%-
%*----------- SLIDE -------------------------------------------------------------
\begin{frame}[t]{Resultado esperado} 
    \transdissolve[duration=0.5]
    %\newline
        \begin{columns}[t]
            \column{.05\linewidth}
            \column{1\linewidth}
                \begin{enumerate}
                    \item Um robô bípede de pequeno porte capaz de se locomover e desviar de obstáculos 
                    \item Artigos científicos que gerarão conhecimento para a sociedade sobre robótica
                    \item Capacitação e formação do orientando na área de robótica
                \end{enumerate}
            \column{.6\linewidth}
        \end{columns}
%*----------- notes
    \note[item]{Notes can help you to remember important information. Turn on the notes option.}
\end{frame}
%-
%*----------- SLIDE -------------------------------------------------------------
\begin{frame}[t]{Principais aprendizados} 
    \transdissolve[duration=0.5]
    %\newline
        \begin{columns}[t]
            \column{.05\linewidth}
            \column{1\linewidth}
                \begin{enumerate}
                    \item Uso de diversos softwares
                    \item Comportamento em ambiente de trabalho
                    \item Importância da organização e planejamento
                    \item Desenvolvimento pessoal 
                    \item Desenvolvimento profissional
                    \item Ciência e Tecnologia são essenciais para a evolução humana
                \end{enumerate}
            \column{.6\linewidth}
        \end{columns}
%*----------- notes
    \note[item]{Notes can help you to remember important information. Turn on the notes option.}
\end{frame}
%-
